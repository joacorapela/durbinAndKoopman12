\begin{lemma}

    Let

    \begin{align}
        p(\mathbf{y}|\mathbf{x})&=\mathcal{N}(\mathbf{y}|A\mathbf{x}+\mathbf{b},\Sigma)\label{eq:pyGx}\\
        p(\mathbf{x})&=\mathcal{N}(\mathbf{x}|\boldsymbol{\mu},\Lambda)\label{eq:px}
    \end{align}

    \noindent then

    \begin{align}
        p(\mathbf{y})=\mathcal{N}(\mathbf{y}|A\boldsymbol{\mu}+\mathbf{b},A\Lambda A^\intercal+\Sigma)\label{eq:py}
    \end{align}

\end{lemma}

\begin{proof}

    \begin{align}
        \ln p(\mathbf{x},\mathbf{y})&=\ln p(\mathbf{y}|\mathbf{x})+\ln p(\mathbf{x})\nonumber\\
                                    &=-\frac{1}{2}\left(\mathbf{y}-\left(A\mathbf{x}+\mathbf{b}\right)\right)^\intercal\Sigma^{-1}\left(\mathbf{y}-\left(A\mathbf{x}+\mathbf{b}\right)\right)-\frac{1}{2}\left(\mathbf{x}-\boldsymbol{\mu}\right)^\intercal\Lambda^{-1}\left(\mathbf{x}-\boldsymbol{\mu}\right)+K_1\nonumber\\
                                    &=-\frac{1}{2}\mathbf{y}^\intercal\Sigma^{-1}\mathbf{y}+\frac{1}{2}\mathbf{y}^\intercal\Sigma^{-1}A\mathbf{x}+\frac{1}{2}\mathbf{x}^\intercal
                                    A^\intercal\Sigma^{-1}\mathbf{y}-\frac{1}{2}\mathbf{x}^\intercal\left(A^\intercal\Sigma^{-1}A+\Lambda^{-1}\right)\mathbf{x}\nonumber\\
                                    &\quad+\frac{1}{2}\mathbf{y}^\intercal\Lambda^{-1}\mathbf{b}+\frac{1}{2}\mathbf{x}^\intercal(-A^\intercal\Sigma^{-1}\mathbf{b}+\Lambda\boldsymbol{\mu})+\frac{1}{2}\mathbf{b}^\intercal\Lambda^{-1}\mathbf{y}+\frac{1}{2}(-\mathbf{b}^\intercal\Sigma^{-1}A+\boldsymbol{\mu}^\intercal\Lambda)\mathbf{x}+K_2\nonumber\\
                                    &=-\frac{1}{2}[\mathbf{x}^\intercal,\mathbf{y}^\intercal]
                                    \left[\begin{array}{cc}
                                        A^\intercal\Sigma^{-1}A+\Lambda^{-1}&-A^\intercal\Sigma^{-1}\nonumber\\
                                        -\Sigma^{-1}A&\Sigma^{-1}
                                    \end{array}\right]\left[\begin{array}{c}
                                                                \mathbf{x}\nonumber\\
                                                                \mathbf{y}
                                                            \end{array}\right]\nonumber\\
                                    &\quad+\frac{1}{2}[\mathbf{x}^\intercal,\mathbf{y}^\intercal]
                                    \left[\begin{array}{c}
                                        A^\intercal\Sigma^{-1}\mathbf{b}+\Lambda^{-1}\boldsymbol{\mu}\\
                                        -\Sigma^{-1}\mathbf{b}
                                    \end{array}\right]+
                                    \frac{1}{2}[\mathbf{b}^\intercal\Sigma^{-1}A+\boldsymbol{\mu}^\intercal\Lambda^{-1},-\mathbf{b}^\intercal\Sigma^{-1}]
                                    \left[\begin{array}{c}
                                        \mathbf{x}\\
                                        \mathbf{y}
                                    \end{array}\right]+K_2\label{eq:completingSquares1}
    \end{align}

    \noindent where $K_1$ and $K_2$ are contants that does not depend on $\mathbf{x}$ or
    $\mathbf{y}$.

    Because $\ln p(\mathbf{x},\mathbf{y})$ is a quadratic form, then
    $p(\mathbf{x},\mathbf{y})$ is a normal probability density function (pdf),
    thus its marginal $p(\mathbf{y})$ is also a normal pdf.
    Our aim is to derive the mean and covariance of $\mathbf{y}$,
    $\boldsymbol{\mu}_y$ and $\Gamma_{yy}$, respectively.

    Call

    \begin{align}
        p(\mathbf{x},\mathbf{y})=\mathcal{N}\left(\left[\begin{array}{c}
                                                            \mathbf{x}\\
                                                            \mathbf{y}
                                                        \end{array}\right]\left|\left[\begin{array}{c}
                                                                                          \boldsymbol{\mu}_x\\
                                                                                          \boldsymbol{\mu}_y\\
                                                                                      \end{array}\right]\right.,\Gamma\right)\label{eq:pxy}
    \end{align}

    \noindent with

    \begin{align*}
        \Phi^{-1}=\Gamma=\left[\begin{array}{cc}
                                   \Gamma_{xx}&\Gamma_{xy}\\
                                   \Gamma_{yx}&\Gamma_{yy}
                               \end{array}\right]
    \end{align*}

    Next,

    \begin{align}
        \ln p(\mathbf{x},\mathbf{y})&=-\frac{1}{2}\left[(\mathbf{x}-\boldsymbol{\mu}_x)^\intercal,(\mathbf{y}-\boldsymbol{\mu}_y)^\intercal\right]\Phi\left[(\mathbf{x}-\boldsymbol{\mu}_x),(\mathbf{y}-\boldsymbol{\mu}_y))\right]+K_1\nonumber\\
                                    &=-\frac{1}{2}[\mathbf{x}^\intercal,\mathbf{y}^\intercal]\Phi\raisebox{-0.7em}{$\left[\begin{array}{c}\mathbf{x}\\\mathbf{y}\end{array}\right]$}+\frac{1}{2}[\mathbf{x}^\intercal,\mathbf{y}^\intercal]\Phi\raisebox{-0.7em}{$\left[\begin{array}{c}\boldsymbol{\mu}_x\\\boldsymbol{\mu}_y\end{array}\right]$}+\frac{1}{2}[\boldsymbol{\mu}_x^\intercal,\boldsymbol{\mu}_y^\intercal]\Phi\raisebox{-0.7em}{$\left[\begin{array}{c}\mathbf{x}\\\mathbf{y}\end{array}\right]$}+K_2\label{eq:completingSquares2}
    \end{align}

    \noindent where $K_1$ and $K_2$ are contants that does not depend on $\mathbf{x}$ or
    $\mathbf{y}$.


    From Eqs.~\ref{eq:completingSquares1} and~\ref{eq:completingSquares2}
    it follows that

    \begin{align}
        \Phi=\left[\begin{array}{cc}
                       A^\intercal\Sigma^{-1}A+\Lambda^{-1}&-A^\intercal\Sigma^{-1}\nonumber\\
                       -\Sigma^{-1}A&\Sigma^{-1}
                   \end{array}\right]
    \end{align}

    \noindent and

    \begin{align*}
        \raisebox{0.7em}{$\Phi$}\left[\begin{array}{c}\boldsymbol{\mu}_x\\\boldsymbol{\mu}_y\end{array}\right]=
            \left[\begin{array}{c}
                      A^\intercal\Sigma^{-1}\mathbf{b}+\Lambda^{-1}\boldsymbol{\mu}\\
                      -\Sigma^{-1}\mathbf{b}
                  \end{array}\right]
    \end{align*}

    Then

    \begin{align}
        \Gamma&=\left[\begin{array}{cc}
                         \Gamma_{xx}&\Gamma_{xy}\\
                         \Gamma_{yx}&\Gamma_{yy}
                     \end{array}\right]=
        \Phi^{-1}=\left[\begin{array}{cc}
                                    A^\intercal\Sigma^{-1}A+\Lambda^{-1}&-A^\intercal\Sigma^{-1}\\
                                    -\Sigma^{-1}A&\Sigma^{-1}
                                \end{array}\right]^{-1}
                        =\left[\begin{array}{cc}
                            \Lambda&\Lambda A\\
                                    A\Lambda&\Sigma+A\Lambda A^\intercal
                                \end{array}\right]\label{eq:Gamma_final}\\
        \left[\begin{array}{c}\boldsymbol{\mu}_x\\\boldsymbol{\mu}_y\end{array}\right]&=
            \Phi^{-1}\left[\begin{array}{c}
                      A^\intercal\Sigma^{-1}\mathbf{b}+\Lambda^{-1}\boldsymbol{\mu}\\
                      -\Sigma^{-1}\mathbf{b}
                  \end{array}\right]=
            \Gamma\left[\begin{array}{c}
                      A^\intercal\Sigma^{-1}\mathbf{b}+\Lambda^{-1}\boldsymbol{\mu}\\
                      -\Sigma^{-1}\mathbf{b}
                  \end{array}\right]
                  =\left[\begin{array}{c}
                              \boldsymbol{\mu}\\
                              A\boldsymbol{\mu}+\mathbf{b}
                          \end{array}\right]\nonumber
    \end{align}

    Thus,

    \begin{align*}
        \Gamma_{yy}&=\Sigma+A\Lambda A^\intercal\\
        \boldsymbol{\mu}_y&=A\boldsymbol{\mu}+\mathbf{b}
    \end{align*}

\end{proof}

Note:

\begin{enumerate}
    \item The last equality in Eq.~\ref{eq:Gamma_final} follows from
        Lemma~\ref{lemma:inverseOfBlockMatrix}.
\end{enumerate}
